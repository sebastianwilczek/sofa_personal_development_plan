\section{Development}
\label{sec:development}
\lhead{\thesection \space Development}

This chapter covers three activities that the author wishes to improve their competence in over the course of the project. Each activity is detailed in the context of one architectural layer, such as \textit{Software} or \textit{User Interaction}. The current level of competence is being described first, followed by the expected improvement during the project.

\subsection{User Interaction Implementation}
\label{ssec:analysis}

The author currently possesses a skill level of Level 1 when it comes to implementing processes in the context of User Interaction. They are able to implement static forms of elements that a user can interact with. To do so, a limited amount of interactive components is being made use of.
\newline
To gain advanced skills, the author would like to learn to make use of innovative technologies to enhance the experience of the user interacting with the developed product. They would like to become familiar with simple yet productive techniques and frameworks to be used in terms of User Interaction. If possible and applicable for the project, the author would also like to familiarize with the possibilities of testing graphical user interfaces and User Interaction in general.

\subsection{Software Design}
\label{ssec:design}

Having worked on multiple different projects before, the author is able to design software systems from scratch. The skill do come up with prototypes and to design both for implementation and tests of new and existing components of a system is given, as specified in Level 2 of the BOKS model.
\newline
The author would like to improve their skill by developing their ability to design complex software systems, consisting of new, existing and perhaps deprecated components. It should be improved how the author can set up a system so that is satisfies both stakeholders and quality assurance. The creation of a test strategy will also be among the to-be-evolved skill set.

\subsection{Software Implementation}
\label{ssec:implementation}

The author is able to build software system, both independent and more complex systems with multiple subcomponents, including the ability to integrate the developed systems into existing software and environments. The ability to perform unit and system tests regarding the performance and integrity of the built products is also present. This is equivalent to Level 2 of the BOKS model.
\newline
A goal of this project is to gain advanced skills in the context of software implementation, equivalent to Level 3. The author wants to learn how to build software on an already designed architecture. The nature of the project is a perfect base for this learning goal, since a main part of the development work revolves around \textit{React Native}, a framework designed to make use of reusable components. Utilizing existing components, both developed by the customer and available through open source, as well as writing and integrating new components is a good practise to reach this learning goal. Gaining skills in test automation is also a skill that should be advanced by participating in the project. However, the often GUI-centric nature of developing with \textit{React Native} may make it hard to utilize test automation to its full extent in this project. The author will try to apply it wherever possible.